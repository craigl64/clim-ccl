% -*- Mode: LaTeX; Package: CLIM-USER -*-

%%
%%  MACROS for CLIM Specs
%%

\newskip  \normalparskip        
\normalparskip = 0pc

\def\removedepth{\ifdim \prevdepth>-1000pt \vskip -\prevdepth\fi}
\def\Vskip #1!{\endgraf
               \removedepth     
               \ifdim \lastskip<#1 \ifdim \lastskip>0pc
               \removelastskip\fi 
               \vskip#1\fi}

%%
%% Basic macros
%%

\def\curly#1{$\{${\it #1\/}$\}$}
\def\star#1{#1{\rm *}}
\def\form{\curly{form}}
\def\place{\curly{place}}
\def\paren#1{\rm({\it #1\/})}
\def\brac#1{\rm[{\it #1\/}]}
\def\ttbrac#1{\tt[{\it #1\/}]}
\def\plus#1{$\hbox{#1}^+$}
\def\placeplus{ $\{${\it place}$\}^+$}
\def\lparen{{\rm (}}
\def\rparen{{\rm )}}

\def\optional{{\tt\&optional\ }}
\def\rest{{\tt\&rest\ }}
\def\key{{\tt\&key\ }}
\def\allow{{\tt\&allow-other-keys\ }}
\def\body{{\tt\&body\ }}

\def\arg#1{{\it #1}}

%% When you supply a default value for optional or keyword arguments,
%% use this idiom:  \key (filled \cl{t})
%% \newcommand {\cl} [1] {{\tt #1}}

%% This allows hyphenation of CL symbols at the hyphens that exist in
%% the symbol name, but noplace else, e.g., \cl{standard\-region\-union}.
%% The reason we don't use this is that the backslashes in the symbol
%% names make Tags Search next to useless.
%% \def\cl#1{{\def\-{\discretionary{-}{}{-}}\tt #1}}

%% Spiffier version that hyphenates only at hyphens, but makes hyphens
%% be active characters, so we don't break Tags Search.  Thanks to
%% Stephen Gildea for providing this.
{\catcode`\-=\active
\gdef\cl{\bgroup
  \catcode`\-=\active \def-{{\tt \char`\-}\penalty\exhyphenpenalty}%
  \@cl}}
\def\@cl#1{\tt#1\egroup}

%% Use \concept when you are introducing a term for the first time, and
%% want it in the index.  Use \term after that.
\newcommand {\concept} [1] {{\sl #1}\index{#1}}
\newcommand {\term} [1] {{\sl #1}}

%%
%% Miscellaneous
%%

\long\def\comment #1 {}

\long\def\keepout #1 {}

\long\def\Issue #1 #2 {{\bf Major issue: } {\sl #2 ---~#1}}
\long\def\issue #1 #2 {{\bf Minor issue: } {\sl #2 ---~#1}}

%%
%% Definition types
%%

\def\outdent#1{\noindent\hbox to 0pc{\hskip -1.5pc #1\hss}\ignorespaces}

\newcommand {\Dodocf} [3] {\outdent{$\Rightarrow$}{\tt #1 {\it #2} \hfill\mbox{\brac{\it #3\/}}}
                           \index{{\tt #1} #3}
                           \Vskip\normalparskip!}
\newcommand {\Dodocv} [2] {\outdent{$\Rightarrow$}{\tt #1 \hfill\mbox{\brac{\it #2\/}}}
                           \index{{\tt #1} #2}
                           \Vskip\normalparskip!}

%% Use these when you want to leave some vertical whitespace afterwards
\def\Defmacro #1 #2 {\Dodocf {#1} {#2} {Macro}}
\def\Defun #1 #2 {\Dodocf {#1} {#2} {Function}}
\def\Defgeneric #1 #2 {\Dodocf {#1} {#2} {Generic~Function}}
\def\Defmethod #1 #2 {\Dodocf {#1} {#2} {Method}}
\def\Defaftermethod #1 #2 {\Dodocf {#1} {#2} {:After~Method}}
\def\Defaroundmethod #1 #2 {\Dodocf {#1} {#2} {:Around~Method}}
\def\Defvar #1 {\Dodocv {#1} {Variable}}
\def\Defconst #1 {\Dodocv {#1} {Constant}}
\def\Defprotoclass #1 {\Dodocv {#1} {Protocol~Class}}
\def\Defpredicate #1 #2 {\Dodocf {#1} {#2} {Protocol~Predicate}}
\def\Defclass #1 {\Dodocv {#1} {Class}}
\def\Definitarg #1 {\Dodocv {#1} {Initarg}}
\def\Deftype #1 {\Dodocv {#1} {Type}}
\def\Defoption #1 {\Dodocv {#1} {Option}}
\def\Defptype #1 #2 {\Dodocf {#1} {#2} {Presentation~Type}}
\def\DefptypeAbbrev #1 #2 {\Dodocf {#1} {#2} {Presentation~Type~Abbreviation}}
\def\Defcondition #1 {\Dodocv {#1} {Condition}}
\def\Deferror #1 {\Dodocv {#1} {Error Condition}}
\def\Defrestart #1 {\Dodocv {#1} {Restart}}

\newcommand {\dodocf} [3] {\outdent{$\Rightarrow$}{\tt #1 {\it #2} \hfill\mbox{\brac{\it #3\/}}} 
                           \index{{\tt #1} #3}
                           \linebreak}
\newcommand {\dodocv} [2] {\outdent{$\Rightarrow$}{\tt #1 \hfill\mbox{\brac{\it #2\/}}}
                           \index{{\tt #1} #2}
                           \linebreak}

%% Use these when you don't want to leave any vertical whitespace afterwards
\def\defmacro #1 #2 {\dodocf {#1} {#2} {Macro}}
\def\defun #1 #2 {\dodocf {#1} {#2} {Function}}
\def\defgeneric #1 #2 {\dodocf {#1} {#2} {Generic~Function}}
\def\defmethod #1 #2 {\dodocf {#1} {#2} {Method}}
\def\defvar #1 {\dodocv {#1} {Variable}}
\def\defconst #1 {\dodocv {#1} {Constant}}
\def\defprotoclass #1 {\dodocv {#1} {Protocol~Class}}
\def\defpredicate #1 #2 {\dodocf {#1} {#2} {Protocol~Predicate}}
\def\defclass #1 {\dodocv {#1} {Class}}
\def\definitarg #1 {\dodocv {#1} {Initarg}}
\def\defoption #1 {\dodocv {#1} {Option}}

\def\Defcommandtable #1 {\Dodocv {#1} {Command~Table}}
\def\Defframe #1 {\Dodocv {#1} {Application~Frame}}

\def\Defgadget #1 {\Dodocv {#1} {Abstract~Gadget}}
\def\defgadget #1 {\dodocv {#1} {Abstract~Gadget}}

\def\Defspane #1 {\Dodocv {#1} {Service~Pane}}
\def\defspane #1 {\dodocv {#1} {Service~Pane}}
\def\Deflpane #1 {\Dodocv {#1} {Layout~Pane}}
\def\deflpane #1 {\dodocv {#1} {Layout~Pane}}

\def\Callback #1 #2 {\Dodocf {#1} {#2} {Callback~Generic~Function}}
\def\callback #1 #2 {\dodocf {#1} {#2} {Callback~Generic~Function}}

%%
%% Canned phrases.
%%

\def\IfYouWantClass #1 #2 #3 {If you want to create a new class that behaves
like #1 #2, it should be a subclass of \cl{#3}.  All instantiable subclasses of
\cl{#3} must obey the #2 protocol.\ }

\def\AbstractClass{This class is an abstract class, intended only to be
subclassed, not instantiated.\ }

\def\Mutable{Members of this class are mutable.\ }

\def\Immutable{Members of this class are immutable.\ }

\def\UncapturedInputs{This function does not capture any of its mutable inputs.\ }

\def\MayCaptureInputs{This function is permitted to capture its mutable inputs; the
consequences of modifying those objects are unspecified.\ }

\def\FreshOutputs{This function returns fresh objects that may be modified.\ } 

\def\ReadOnly{This function returns objects that reveal CLIM's internal state;
do not modify those objects.\ }

\def\ReadWrite{This function returns objects that reveal CLIM's internal state;
these objects may be modified.\ }

\def\NotInRelease2{{\sl This is not fully supported in Release 2.\ }} 
